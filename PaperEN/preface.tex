\chapter*{Introduction}
\addcontentsline{toc}{chapter}{Introduction}

Nature tends to behave in a symmetrical way. This can be seen on many examples, ranging from petal symmetries of flowers to the symmetric structure of lattices of crystalline solids. For this reason, symmetrical objects were studied heavily during the history of humankind. One such class of objects are the Platonic and Archimedean solids.

Since both of these classes fall into the category of convex polyhedra, we can use projections on these solids to obtain their graphs called Platonic and Archimedean graphs respectively. This allows us to formulate questions about the underlying solids, by the means of a relatively young field of mathematics called graph theory.

This brings is to the topic of this thesis, which will concern itself with various colorings of Platonic and Archimedean graphs. First we will provide an overview of known facts about their chromatic numbers for traditional colorings, (e.g. vertex coloring, edge coloring) which have been already been studied and discovered. Later, we will focus on more unconventional types of colorings, where still some interesting revelations can be made.

\section{Platonic solids}

Let us first define what objects we will be coloring. Luckily, we are going to be working only with \emph{convex} polyhedra, which are easier to define than general polyhedra. In addition, when we only restrict ourselves to 3-dimensions, these objects have a nice property proved by Steinitz, that is, they exactly correspond to 3-vertex-connected planar graphs.

Moreover, the solids we are considering are also regular or semi-regular.

\begin{define}
For $A\in\mathbb{R}^{n \times m}$, $b\in\mathbb{R}^n$ {\sl convex polyherdron} is the set:
\[
\{ x \in \mathbb{R}^m | Ax \leq b\}
\]
\end{define}

\begin{define}
A~{\sl Platonic solid} is a connected graph with no cycles.
\end{define}
