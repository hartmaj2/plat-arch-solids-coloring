\chapter{Definitions}

Since the main concern of this thesis is coloring of the graphs mentioned in the preface, it is useful to define all the necessary concepts we will be working with.

\section{Graph theory}

\begin{definition}
    An \textit{undirected graph} $G$ is an ordered pair $G=(V,E)$ where $V$ is a set of vertices of the graph and $E \subseteq \binom{V}{2}$ is the set of its edges.
\end{definition}

\section{Typical colorings}

\begin{definition}
    A \textit{vertex coloring} of a graph is a function $c: V \rightarrow \mathbb{N}$ s.t. $\forall \{u,v\}\in E : c(u) \neq c(v)$
\end{definition}

In other words, coloring is an assignment of colors, represented by natural numbers, to each vertex with the following condition: If two vertices are adjacent, they must be of different color.

\begin{figure}[H]
    \centering
    \includegraphics[width=0.2\textwidth]{../Resources/Figs/octahedral_vtx_colr.pdf}
    \caption{Vertex coloring of octahedral graph}
    \label{fig:octahedral_vtx_coloring}
\end{figure}

There are more types of colorings that can be considered, but for every coloring, we are usually interested in the so called \textit{chromatic number} of the given graph under the given coloring.

\begin{definition}
    For a graph $G=(V,E)$ and some coloring $c$, the \textit{chromatic number} $\chi(G)$ is the minimum number $n$, s.t. $G$ can be colored using $n$ colors.
\end{definition}

We can also consider colorings of edges.

\begin{definition}
    An \textit{edge coloring} of a graph is a function $c: E \rightarrow \mathbb{N}$ s.t. $\forall e,f \in E : e \cap f \neq \emptyset \implies c(e) \neq c(f)$
\end{definition}

\begin{figure}[H]
    \centering
    \includegraphics[width=0.2\textwidth]{../Resources/Figs/cubical_edg_colr.pdf}
    \caption{Edge coloring of cubical graph}
    \label{fig:cubical_edge_coloring}
\end{figure}

\section{Special colorings}