\chapter{Number of colorings up to symmetries}
\label{chap:clrings-up-to-symmetries}

In the previous chapters, we computed, how many proper colorings of Platonic and Archimedean solids exists using the notion of the chromatic polynomial. In that approach, we didn't take into account, that some colorings can be obtained from the other ones by simply rotating or reflecting the solid i.e. by applying some geometric transformation on it. When one coloring can be obtained from another one using some geometric transformation, then it makes good sense to regard these two colorings as identical. An analogy from chemistry is, that the properties of a molecule do not change, when we rotate it. Thus it is reasonable, to consider all its rotations as being the same molecule. 

In order to be able to say, which colorings should be considered identical, we need to define some kind of equivalence relation, that will group all the different transformations of same coloring together. In mathematics, we call this relation a \textit{transitivity relation} and its equivalence classes are typically referred to as its \textit{orbits}. The transitivity relation is defined on a set of objects, in our case the colorings, and a group that acts on these objects, the transformations.

Using the terminology above, our task boils down to counting the number of orbits of the transitivity relation, defined on colorings of our solid using the group of transformations of the solid.

To be able to use the transformations of the solids in our computations, we need a mathematical way of describing them. As we are working with graphs of the solids, we then also need to be able to describe these transformations as transformations on these graphs.


\section{Preliminaries}

It turns out, that transformations of the solid that we consider correspond exactly to automorphisms of the corresponding graph of the solid.

\begin{defn}[automorphism]
    Let $G=(V,E)$ be a graph. Then a function $f:V \rightarrow V$ is an \emph{automorphism} of $G$ if it is a bijection and for every $\{u,v\} \in E$ we have that $\{f(u),f(v)\} \in E$.
\end{defn}

For a graph $G$, we will denote the set of all automorphisms of $G$ by $\Aut(G)$.

\begin{defn}(group)
    A \emph{group} is a pair $\mathcal{G}=(G,\circ)$ where $G$ is a set, $\circ:G^2 \rightarrow G$ a binary operation s.t. the following hold:
    \begin{enumerate}
        \item $\forall x,y,z \in G : x \circ (y \circ z) = ( x \circ y ) \circ z$
        \item $\exists e \in G \ \forall x \in G :x \circ e = x = e \circ x$
        \item $ \forall x \in G \ \exists y \in G : x \circ y = e = y \circ x$
    \end{enumerate}
\end{defn}

An important observation is, that for a graph $G$, the set of all automorphisms $\Aut(G)$ together with the function composition operation forms a group. 

Before we define can define the transitivity relation, we also need to properly define the set of elements on which this relation will be defined:

\begin{defn}[colorings with up to n colors]
    For a graph $G$, positive integer $n$ and a family of colorings $C$, we define the set $C_n \subseteq C$ the set of all proper colorings of $G$ by at most $n$ colors.
\end{defn}

The definition above restricts the colorings we will be working with to finite sets only. With the notion of automorphism and colorings in hand, we can define the transitivity relation for the colorings.

\begin{defn}(transitivity relation)
        Let $G=(V,E)$ be a graph and $n\in \mathbb{N}$. Let $c_1,c_2 \in C_n$. We define $c_1 \sim c_2$ whenever exists an automorphism $\alpha \in \Aut(G) : \forall v \in V : c_1(v)=c_2(\alpha(v))$.
\end{defn}

Less formally, two colorings are related, when we can find an automorphism that maps vertices of any color in the first coloring, to vertices of the same color in the second coloring. 

We can observe that for any coloring $c$ of $G$, we have $c \sim c$ by simply taking the identity automorphism $id \in \Aut(G)$. So the transitivity relation is \textit{reflexive}. It is also \textit{symmetric} by using the fact, that if $c_1 \sim c_2$ by existence of automorphism $\alpha$, then we can use $\alpha^{-1}$ to show that $c_2 \sim c_1$ as well. Lastly, we can also show that the relation $\sim$ is \textit{transitive}. This comes from the fact, that composing two automorphisms $\alpha$ and $\psi$ yields an automorphism. So in fact, the transitivity relation defined above is an eqivalence relation.

\begin{defn}(orbit)
    For $n \in \mathbb{N}$ and a coloring $c \in C_n$ of some graph $G$ we will call the eqivalence class $[c]_{\sim}$ an \emph{orbit} of $c$ and denote it by $\Orb(c)$.
\end{defn}

\begin{defn}(set of all orbits)
    For $n \in \mathbb{N}$ and a set $C_n$ of proper colorings of graph $G$, we denote $C_n/_\sim$ the set of all eqivalence classes of the transitivity relation $\sim$ on $C_n$.
\end{defn}

Here is a good place to point out, that it is exactly our objective, to calculate the size of this set, $\abs{C_n/_\sim}$.

\begin{defn}(stabilizer)
    For $n \in \mathbb{N}$ and a coloring $c \in C_n$ of $G=(V,E)$, we will call the set $\Stab(c) = \{ \alpha \in \Aut(G) \ | \ \forall v \in V : c(\alpha(v)) = c(v) \}$ a \emph{stabilizer} of $c$. 
\end{defn}

Again, to say more coloquially, the stabilizer is the set of all automorphisms that when applied to a coloring will result in exactly the same coloring. What we mean by same colorings is, that all the vertices get the same color under both colorings.

\section{Calculating the number of orbits}

Now it can be shown using the Lagrange's theorem that $\abs{\Stab(c)} \cdot \abs{\Orb(c)} = \abs{\Aut(G)}$ which is the key fact that allows us to be able to compute the number of equivalence classes of the transitivity relation. Mathematically formulated, it allows us to derive the following equation:

\begin{equation}\label{eqn:orbits-by-stab}
    \sum_{c\in C_n}\abs{\Stab(c)} = \abs{\Aut(G)} \cdot \abs{C_n/_\sim} 
\end{equation}


Usually, this formula is not suitable for practical calculations because the size of set $C_n$ can be quite large. In most cases, the set $\Aut(G)$ is much smaller and thus it is more reasonable, to sum over its elements instead. 

For example, consider the graph of cube $G_{cube}$. Taking for $C_4$ the set of all proper colorings of $G_{cube}$ by at most $4$ colors, we get $\abs{C_4} = 2652$. This value can be obtained by plugging number $4$ to the chromatic polynomial of $G_{cube}$. On the other hand, by consulting table \ref{tab:plat-automorphisms} below, we have $\abs{\Aut(G_{cube})} = 48$ which is more than hundred times smaller number.

Before stating another important observation, we will need a following definition:

\begin{defn}[fixpoint]
    For a graph $G$, $n \in \mathbb{N}$, a set of its proper colorings $C_n$ and automorphism $\alpha \in \Aut(G)$ we denote $\Fix(\alpha)= \{ c \in C \ | \ \forall v \in V : c(\alpha(v)) = c(v) \}$
\end{defn}

To get a formula which computes the same number but using a sum over the set $\Aut(G)$ instead, we use the following observation: The sum $\sum_{c \in C_n} \abs{\Stab(c)}$ computes the number of ordered pairs $(c,\alpha)$ s.t. $\forall v \in V : c(\alpha(v)) = c(v)$. Using the definition of fixpoint, we can calculate this number also using the sum $\sum_{\alpha \in \Aut(G)} \abs{\Fix(\alpha)}$ which yields the following important identity:
\begin{equation}\label{eqn:two-way-counting}
    \sum_{\alpha \in \Aut(G)} \abs{\Fix(\alpha)} = \sum_{c \in C_n}\abs{\Stab(c)}    
\end{equation}

By combining equations \ref{eqn:orbits-by-stab} and \ref{eqn:two-way-counting}, we get the following result which is a direct application of the famous \textit{Burnside's lemma} on our particular setting:

\begin{thm}[application of Burnside's lemma] \label{thm:burnside}
    For a graph $G$, $n \in \mathbb{N}$ and a set of its proper colorings $C_n$, we have:
    $$\abs{C_n/_\sim} = \frac{1}{\abs{\Aut(G)}}\cdot \sum_{\alpha \in \Aut(G)}\abs{\Fix(\alpha)}$$
\end{thm}

