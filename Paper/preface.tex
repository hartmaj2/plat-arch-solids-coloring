\chapter*{Introduction}
\addcontentsline{toc}{chapter}{Introduction}

Nature tends to behave in a symmetrical way. This can be seen on many examples, ranging from petal symmetries of flowers to the symmetric structure of lattices of crystalline solids. For this reason, symmetrical objects were studied heavily during the history of humankind. One such class of objects are the Platonic and Archimedean solids.

Since both of these classes fall into the category of convex polyhedra, projections of these solids on a sphere can be used to obtain their graphs called Platonic and Archimedean graphs respectively. This allows us to formulate questions about the underlying solids, by the means of a relatively young field of mathematics called graph theory.

This brings us to the topic of this thesis, which concerns itself with various colorings of Platonic and Archimedean graphs. First we provide an overview of known facts about the solids' chromatic numbers for traditional colorings, (e.g. vertex coloring, edge coloring) which have been already been studied and discovered. Later, we focus on more unconventional types of colorings, where still some interesting revelations can be made.

\section{Platonic and Archimedean solids}

Platonic solids are regular \textbf{convex} polyhedra. Regular polyhedron is a polyhedron whose all faces are congruent to a single regular polygon i.e. a polygon with all sides of equal length. There are exactly five Platonic solids: tetrahedron, cube, octahedron, dodecahedron and icosahedron.

Note that convexity is an important property for us, since it ensures, that we can always obtain a 3-vertex connected planar graph. This fact is also known as Steinitz’ theorem for polyhedra \cite{kendall24}.

The Archimedean solids can then be obtained by performing one of the following non-disjoint operations on Platonic solids:

\begin{description}
    \item[Truncation] Removes corners of the polyhedron by making a cut perpendicular to a line connecting the vertex of the corresponding corner with the centroid of the polyhedron.
    \item[Rectification] Special case of truncation where the truncating cuts are done in such way, that they go through midpoints of the edges connecting the vertices. 
    \begin{figure}[H]
        \centering
        \includegraphics[width=1\textwidth]{../Resources/Figs/truncation.pdf}
        \caption{Trunctation and rectification operations visualised \cite{wikimedia-cube-truncation}}
        \label{fig:truncation}
    \end{figure}
    \item[Expansion/Cantellation] All the facets are pulled out away from the centroid of the polyhedron by the same distance without rescaling. The empty spaces are then filled with regular polygons in the following way: Edges that used to be identical in the original polyhedron are connected by adding a new a square. All the vertices that corresponded to a single vertex $v$ in the original polyhedron are connected by adding a $d$-gon where $d=deg(v)$.

    This operation can also be described as cantellation. The difference is in how we imagine the process that takes us to the resulting shape. From the viewpoint of cantellation, we first bevel (cut off) the edges and then apply truncation on what remained from the original vertices.
        \begin{figure}[H]
        \centering
        \includegraphics[width=1\textwidth]{../Resources/Figs/cantellation.pdf}
        \caption{Cantellation operation visualised \cite{wikimedia-cube-cantellation}}
        \label{fig:cantellation}
    \end{figure}
    \item[Snub] Is an application of expansion followed by splitting each new square in half in such a way, that we can twist the facets of the original Platonic solid.
\end{description}


\todo{TODO: Add visualisation of snub operation}


