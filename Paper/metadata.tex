%%% Please fill in basic information on your thesis, which will be automatically
%%% inserted at the right places. You need to replace \xxx{...} by real data.

% Type of your thesis:
%	"bc" for Bachelor's
%	"mgr" for Master's
%	"phd" for PhD
%	"rig" for rigorosum
\def\ThesisType{bc}

% Language of your study programme:
%	"cs" for Czech
%	"en" for English
\def\StudyLanguage{cs}

% Thesis title in English (exactly as in the official assignment)
% (Note: \xxx is a "ToDo label" which makes the unfilled visible. Remove it.)
\def\ThesisTitle{Coloring of Platonic and Archimedean solids}

% Author of the thesis (you)
\def\ThesisAuthor{Jan Hartman}

% Year when the thesis is submitted
\def\YearSubmitted{2025}

% Name of the department or institute, where the work was officially assigned
% (according to the Organizational Structure of MFF UK in English,
% see https://www.mff.cuni.cz/en/faculty/organizational-structure,
% or a full name of a department outside MFF)
\def\Department{Department of Applied Mathematics}

% Is it a department (katedra), or an institute (ústav)?
\def\DeptType{Department} 

% Thesis supervisor: name, surname and titles
\def\Supervisor{doc. RNDr. Jiří Fiala, Ph.D.}

% Supervisor's department (again according to Organizational structure of MFF)
\def\SupervisorsDepartment{Department of Applied Mathematics}

% Study programme (does not apply to rigorosum theses)
\def\StudyProgramme{Computer Science}

% An optional dedication: you can thank whomever you wish (your supervisor,
% consultant, who provided you with tea and pizza, etc.)
\def\Dedication{
I would like to give many thanks to my supervisor, Jiří Fiala, especially for his positive and proactive attitude. I am grateful for all helpful insights and recommendations he gave me throughout his supervision of this thesis.

Nevertheless, nothing of this would be possible without my family and friends, who kindly respected, that I did not have as much time for them as I wished.

I would also like to thank developers of the SageMath software. They did and are still doing great work on a software, that provided me with tools that I could use for most of my algorithms, saving me from having to reinvent the wheel.

Lastly, I would like to thank Martin Mareš and others, who took part in creating the Latex template that was ready to be used for this thesis.
}

% Abstract (recommended length around 80-200 words; this is not a copy of your thesis assignment!)
\def\Abstract{
In this thesis, we study the coloring of Platonic and Archimedean solids. We provide an overview of the properties of their underlying graphs, followed by a summary of the types of colorings that can be applied to these graphs. We show conversions between various types of colorings and compute the corresponding chromatic numbers. We study chromatic polynomials and derive an explicit formula for the chromatic polynomial of a complete k-partite graph with partition size 2. We then study the concept of the orbital chromatic polynomial, which was first introduced by P.J. Cameron in 2007, and implement an algorithm for its computation. Lastly, we study the number of partitions of vertices into independent sets up to symmetries, establish bounds for these numbers, and propose an enumerative algorithm for their computation.}

% \xxx{Use the most precise, shortest sentences that state what problem the
% thesis addresses, how it is approached, pinpoint the exact result achieved, and
% describe the applications and significance of the results. Highlight anything
% novel that was discovered or improved by the thesis. Maximum length is 200
% words, but try to fit into 120. Abstracts are often used for deciding if
% a reviewer will be suitable for the thesis; a well-written abstract thus
% increases the probability of getting a reviewer who will like the thesis.}


% 3 to 5 keywords (recommended) separated by \sep
% Keywords are useful for indexing and searching for the theses by topic.
\def\ThesisKeywords{
graph coloring \sep chromatic polynomial \sep orbit-counting \sep independent sets \sep Platonic solids \sep Archimedean solids
}
% \xxx{keyword\sep key phrase}

% If any of your metadata strings contains TeX macros, you need to provide
% a plain-text version for use in XMP metadata embedded in the output PDF file.
% If you are not sure, check the generated thesis.xmpdata file.
\def\ThesisAuthorXMP{\ThesisAuthor}
\def\ThesisTitleXMP{\ThesisTitle}
\def\ThesisKeywordsXMP{\ThesisKeywords}
\def\AbstractXMP{\Abstract}

% If your abstracts are long and do not fit in the infopage, you can make the
% fonts a bit smaller by this setting. (Also, you should try to compress your abstract more.)
\def\InfoPageFont{}
%\def\InfoPageFont{\small}  % uncomment to decrease font size

% If you are studing in a Czech programme, you also need to provide metadata in Czech:
% (in English programmes, this is not used anywhere)

\def\ThesisTitleCS{Barvení platónských a archimédovských těles}
\def\DepartmentCS{Katedra aplikované matematiky}
\def\DeptTypeCS{Katedra}
\def\SupervisorsDepartmentCS{Katedra aplikované matematiky}
\def\StudyProgrammeCS{Informatika}

\def\ThesisKeywordsCS{
barvení grafů \sep chromatický polynom \sep počítání orbit \sep nezávislé množiny \sep platónská tělesa \sep archimédovská tělesa
}

\def\AbstractCS{
V této práci studujeme obarvení platónských a archimédovských těles. Poskytujeme přehled vlastností jejich grafů, následovaný shrnutím typů obarvení, která lze na tyto grafy aplikovat. Ukazujeme převody mezi různými typy obarvení a počítáme odpovídající chromatická čísla. Studujeme chromatické polynomy a odvozujeme explicitní vzorec pro chromatický polynom úplného k-partitního grafu s partitami velikosti 2. Dále se zabýváme konceptem orbitálního chromatického polynomu, který poprvé představil P. J. Cameron v roce 2007, a implementujeme algoritmus pro jeho výpočet. Nakonec studujeme počet rozkladů vrcholů na nezávislé množiny až na symetrie, stanovujeme odhady pro tyto počty a navrhujeme algoritmus pro jejich výpočet.}
