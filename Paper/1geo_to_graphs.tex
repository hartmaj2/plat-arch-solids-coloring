\chapter{Geometry and basic facts}

\section{Platonic and Archimedean solids}

Platonic solids are regular \textbf{convex} polyhedra. Regular polyhedron is a polyhedron whose all faces are congruent to a single regular polygon i.e. a polygon with all sides of equal length. There are exactly five Platonic solids: tetrahedron, cube, octahedron, dodecahedron and icosahedron.

Note that convexity is an important property for us, since it ensures, that we can always obtain a 3-vertex connected planar graph. This fact is also known as Steinitz’ theorem for polyhedra \cite{kendall24}.

The Archimedean solids can then be obtained by performing one of the following non-disjoint operations on Platonic solids:

\begin{description}
    \item[Truncation] Removes corners of the polyhedron by making a cut perpendicular to a line connecting the vertex of the corresponding corner with the centroid of the polyhedron.
    \item[Rectification] Special case of truncation where the truncating cuts are done in such way, that they go through midpoints of the edges connecting the vertices. 
    \begin{figure}[H]
        \centering
        \includegraphics[width=1\textwidth]{../Resources/Figs/op_truncation.pdf}
        \caption{Trunctation and rectification of cube visualised \cite{wikimedia-cube-truncation}}
        \label{fig:op_truncation}
    \end{figure}
    \item[Expansion/Cantellation] All the facets are pulled out away from the centroid of the polyhedron by the same distance without rescaling. The empty spaces are then filled with regular polygons in the following way: Edges that used to be identical in the original polyhedron are connected by adding a new a square. All the vertices that corresponded to a single vertex $v$ in the original polyhedron are connected by adding a $d$-gon where $d=deg(v)$.

    This operation can also be described as cantellation. The difference is in how we imagine the process that takes us to the resulting shape. From the viewpoint of cantellation, we first bevel (cut off) the edges and then apply truncation on what remained from the original vertices.
    \begin{figure}[H]
        \centering
        \includegraphics[width=1\textwidth]{../Resources/Figs/op_cantellation.pdf}
        \caption{Cube cantellation visualised \cite{wikimedia-cube-cantellation}}
        \label{fig:op_cantellation}
    \end{figure}
    \item[Snub] Is an application of expansion/cantellation followed by splitting each new square in half in such a way, that we can twist the facets of the original Platonic solid.
    \begin{figure}[H]
        \centering
        \includegraphics[width=1\textwidth]{../Resources/Figs/op_snub.pdf}
        \caption{The twisting part of snub operation on cube visualised \cite{natal-polyhed-viewer}}
        \label{fig:op_snub}
    \end{figure}
    
\end{description}

\section{Basic definitions and assumptions}

Here we state mathematical definitions, that should not be surprising in any way i.e. can be considered standard. Also, we state any assumptions we will make, which will then hold for the rest of this thesis.

\begin{defn}[undirected graph]
    An \emph{undirected graph} $G$ is an ordered pair $G=(V,E)$ where $V$ is a set of vertices of the graph and $E \subseteq \binom{V}{2}$ is the set of its edges. 
\end{defn}

Since we are interested in Platonic and Archimedean solids, whose graphs are all planar, it is useful to define the notion of a \textit{plane graph}. This notion will correspond to a drawing of a particular graph onto a plane with one special property: the edges as they are drawn do not cross. 

We use definition of a \textit{drawing} of a graph from the book \textit{Invitation to Discrete Mathematics} by Jaroslav Nešetřil and Jiří Matoušek \cite{matousek2009}. A drawing of a graph might split the plane into multiple separate segments. We call these segments the \textit{faces} of a drawing of graph $G$.

\begin{defn}[planar graph]
    A graph $G$ is \emph{planar} if there exists a drawing of $G$ into a plane.
\end{defn}

\begin{defn}[plane graph]
    A \emph{plane graph} $G' = (V,E,F)$ is a drawing of a planar graph $G=(V,E)$ into plane where $F$ is the set of all faces of this drawing. We will also use the notation $V(G'), E(G'), F(G')$ to refer to sets $V,E,F$ respectively.
\end{defn}

In this thesis, we assume for any plane graph $G=(V,E,F)$ that the sets $V$, $E$ and $F$ are pairwise disjoint.

\section{Platonic and Archimedean graph properties}

Here we would like to list some properties, of the graphs corresponding to the solids described above. Let $G=(V,E,F)$ be a plane graph of a solid with faces $F$. In the following tables, $d$ is a number s.t. $\forall u \in V : \deg(u) = d$. 

The last column is a so called \textit{vertex configuration}, which denotes the number of sides that the polygons incident with each vertex have. Note that the numbers appear in the configuration exactly as the corresponding faces appear around the vertex. For example, the configuration $c_1 = 3.4.3.4$ is different from a hypothetical configuration $c_2 = 3.4.4.3$ even though the numbers $3$ and $4$ appear exactly twice in both cases. In $c_1$, each triangle is surrounded by two squares and vice versa. In $c_2$, each triangle or square has both a triangle and a square next to it. 

\begin{table}[H]
\centering
\begin{tabular}{l@{\hspace{1.5cm}}ccccc}
\toprule
\textbf{Platonic} & \textbf{$|V|$} & \textbf{$|E|$} & \textbf{$|F|$} & \textbf{$d$} & \textbf{Vertex config.} \\
\midrule
tetrahedron & 4 & 6 & 4 & 3 & 3.3.3 \\
octahedron & 6 & 12 & 8 & 4 & 3.3.3.3 \\
cube & 8 & 12 & 6 & 3 & 4.4.4 \\
icosahedron & 12 & 30 & 20 & 5 & 3.3.3.3.3 \\
dodecahedron & 20 & 30 & 12 & 3 & 5.5.5 \\
\bottomrule
\end{tabular}
\caption{Basic properties of Platonic graphs}
\label{tab:platonic-basic-props}
\end{table}

\begin{table}[H]
\centering
\begin{tabular}{l@{\hspace{1.5cm}}ccccc}
\toprule
\textbf{Archimedean} & \textbf{$|V|$} & \textbf{$|E|$} & \textbf{$|F|$} & \textbf{$d$} & \textbf{Vertex config.} \\
\midrule
truncated tetrahedron & 12 & 18 & 8 & 3 & 3.6.6 \\
cuboctahedron & 12 & 24 & 14 & 4 & 3.4.3.4 \\
truncated cube & 24 & 36 & 14 & 3 & 3.8.8 \\
truncated octahedron & 24 & 36 & 14 & 3 & 4.6.6 \\
rhombicuboctahedron & 24 & 48 & 26 & 4 & 3.4.4.4 \\
snub cube & 24 & 60 & 38 & 5 & 3.3.3.3.4 \\
icosidodecahedron & 30 & 60 & 32 & 4 & 3.5.3.5 \\
truncated cuboctahedron & 48 & 72 & 26 & 3 & 4.6.8 \\
truncated icosahedron & 60 & 90 & 32 & 3 & 5.6.6 \\
truncated dodecahedron & 60 & 90 & 32 & 3 & 3.10.10 \\
rhombicosidodecahedron & 60 & 120 & 62 & 4 & 3.4.5.4 \\
snub dodecahedron & 60 & 150 & 92 & 5 & 3.3.3.3.5 \\
truncated icosidodecahedron & 120 & 180 & 62 & 3 & 4.6.10 \\
\bottomrule
\end{tabular}
\caption{Basic properties of Archimedean graphs}
\label{tab:archimedean-basic-props}
\end{table}