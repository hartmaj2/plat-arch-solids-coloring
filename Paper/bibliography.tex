%%% Seznam použité literatury je zpracován podle platných standardů. Povinnou citační
%%% normou pro bakalářskou práci je ISO 690. Jména časopisů lze uvádět zkráceně, ale jen
%%% v kodifikované podobě. Všechny použité zdroje a prameny musí být řádně citovány.

\def\bibname{Bibliography}
\begin{thebibliography}{99}
\addcontentsline{toc}{chapter}{\bibname}

% PAPERS

\bibitem{lamport94}
  {\sc Lamport,} Leslie.
  \emph{\LaTeX: A Document Preparation System}.
  2. vydání.
  Massachusetts: Addison Wesley, 1994.
  ISBN 0-201-52983-1.

\bibitem{kendall24}
  {\sc Kendall,} Matthew.
  \emph{Steinitz’ Theorem for Polyhedra}. Available online at: \\ \url{https://matthewkendall.github.io/assets/steinitz.pdf}. \\
  Accessed: January 8, 2025.

% IMAGES

\bibitem{wikimedia-cube-truncation}
  {\sc Ruen, Tom.} 
  \emph{Cube truncation sequence}. 
  Wikimedia Commons. Available online at: \\ \url{https://commons.wikimedia.org/wiki/File:Cube_truncation_sequence.svg}. \\
  Accessed: February 17, 2025.

\bibitem{wikimedia-cube-cantellation}
  {\sc Ruen, Tom.} 
  \emph{Cube cantellation sequence}. 
  Wikimedia Commons. Available online at: \\ \url{https://commons.wikimedia.org/wiki/File:Cube_cantellation_sequence.svg}. \\
  Accessed: February 17, 2025.

% USED 3RD PARTY CODE

\bibitem{sagemath-chromatic-number}
  {\sc The SageMath Project.} 
  \emph{Chromatic number function}. 
  SageMath Documentation. Available online at: \\ 
  \url{https://doc.sagemath.org/html/en/reference/graphs/sage/graphs/graph_coloring.html#sage.graphs.graph_coloring.chromatic_number}. \\
  Accessed: February 22, 2025.

\bibitem{sagemath-chromatic-index}
  {\sc The SageMath Project.} 
  \emph{Edge coloring function}. 
  SageMath Documentation. Available online at: \\ 
  \url{https://doc.sagemath.org/html/en/reference/graphs/sage/graphs/graph_coloring.html#sage.graphs.graph_coloring.edge_coloring}. \\
  Accessed: February 22, 2025.

\end{thebibliography}
