\chapter*{Introduction}
\addcontentsline{toc}{chapter}{Introduction}

This thesis aims to research various coloring problems on a specific class of graphs, corresponding to graphs of Platonic and Archimedean solids. Since these graphs are highly symmetric, we are especially interested in solving the problems that we introduce with respect to symmetries of these graphs.

By providing an overview of other, less traditional types of colorings, we aim to equip ourselves with a wide range of interesting problems, which we can then study up to symmetries. Our goal is also to demonstrate possible conversions between certain types of colorings in order to convince the reader that, for many cases, it is enough to solve only one particular problem without having to explicitly solve the other instances, which can be obtained by a corresponding conversion. On the other hand, we also describe other coloring problems where no trivial conversions exist, e.g., rainbow coloring or magic labeling.

Nevertheless, the main objective of the thesis is to study colorings of Platonic and Archimedean solids with respect to their symmetries. We investigate the concept of the \emph{orbital chromatic polynomial}, which is a counterpart to the standard chromatic polynomial, and implement an algorithm for computing it, later demonstrating our results on Platonic solids. Here we also aim to compare the difference between the two concepts, for which symmetric graphs such as those of Platonic and Archimedean solids are ideal.

On the other hand, we show that even when we take symmetries into account, the results have their limitations when our goal is to count only structurally different colorings. This leads to the problem of counting partitions of the vertex set into independent sets up to symmetries, for which we provide a partial answer obtained by computer calculation.