\chapter*{Introduction}
\addcontentsline{toc}{chapter}{Introduction}

\begin{highlight}
Main concern of this thesis is to study various coloring problems on a specific class of graphs, corresponding to graphs of Platonic and Archimedean solids. Since these graphs are highly symmetric, we are especially interested in solving the problems, which we introduce, with respect to symmetries of these graphs.

Firstly, we introduce Platonic and Archimedean solids and investigate properties of their graphs. This allows us later, to interpret results of our computations, as they are dependent on those properties. We then research various kinds of colorings in their original form, without taking symmetries into account. In doing so, we equip ourselves with a wide range of interesting problems, which we can then study up to symmetries. 

For many of the coloring problems introduced, there exist conversions between them. Showing these conversion, we would like to convince the reader, that for many cases, it is enough to solve only one particular problem, e.g. the vertex coloring, without having to explicitly solve the other instances, which can be obtained by a corresponding conversion. On the other hand, we also described other coloring problems, where no trivial conversions exist, e.g. rainbow coloring or magic labeling.

After providing an overview of the studied concepts, we get to the main topic of the thesis. We study the chromatic polynomial with respect to symmetries, which corresponds to a relatively young concept of the orbital chromatic polynomial, introduced in 2007 by P.J. Cameron. We implement an algorithm proposed by P.J. Cameron and test it on graph of Platonic and Archimedean solids. We compare chromatic polynomials with orbital chromatic polynomials to get a sense of how much symmetries influence the results.

Lastly, we study the problem of counting the number of graph colorings from a point of view, when we are only interested in partitions of the set of vertices into given number of independent sets. This is an interesting problem, since as of writing this thesis, we do not know of any way to calculate this number, without having to resort to enumerating all colorings and then filtering the ones that should not be counted. We first provide simple bounds on numbers, one should expect to obtain and then test these bounds using an enumerative algorithm. We then illustrate the results we obtain, by showing examples on Platonic and Archimedean solids.
\end{highlight}