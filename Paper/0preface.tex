\chapter*{Introduction}
\addcontentsline{toc}{chapter}{Introduction}

Nature tends to behave in a symmetrical way. This can be seen on many examples, ranging from petal symmetries of flowers to the symmetric structure of lattices of crystalline solids. For this reason, symmetrical objects were studied heavily during the history of humankind. One such class of objects are the Platonic and Archimedean solids.

Since both of these classes fall into the category of convex polyhedra, projections of these solids on a sphere can be used to obtain their graphs called Platonic and Archimedean graphs respectively. This allows us to formulate questions about the underlying solids, by the means of a relatively young field of mathematics called graph theory.

This brings us to the topic of this thesis, which concerns itself with various colorings of Platonic and Archimedean graphs. First we provide an overview of known facts about the solids' chromatic numbers for traditional colorings, (e.g. vertex coloring, edge coloring) which have been already been studied and discovered. Later, we focus on more unconventional types of colorings, where still some interesting revelations can be made.