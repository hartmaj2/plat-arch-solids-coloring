\chapter{Calculations}

\section{Computed selected chromatic numbers}

The vertex and edge chromatic numbers can be computed in little time using \textit{SageMath} functions \cite{sagemath-chromatic-number} \cite{sagemath-chromatic-index}. Since as of time of writing this thesis, there exists no function for directly computing the total chromatic number $\chi''(G)$, the conversion to total graph mentioned in chapter \ref{chap:clring_conversions} must be used. 

The following two tables provide overview of vertex, edge and total chromatic numbers, denoted by $\chi(G)$, $\chi'(G)$ and $\chi''(G)$ respectively, for Platonic and Archimedean solids. Note that the edge chromatic number $\chi'(G)$ is also called the \textit{chromatic index}.

\begin{table}[H]
\centering
\caption{Vertex and edge chromatic numbers of Platonic graphs}
\vspace{5pt}
\label{tab:platonic-chrom-nums}
\begin{tabular}{|l|c|c|c|}
\hline
Platonic & $\chi(G)$ & $\chi'(G)$ & $\chi''(G)$ \\
\hline\hline
cube & 2 & 3 & 4 \\
\hline
dodecahedron & 3 & 3 & 4 \\
\hline
icosahedron & 4 & 5 & 6 \\
\hline
octahedron & 3 & 4 & 5 \\
\hline
tetrahedron & 4 & 3 & 5 \\
\hline
\end{tabular}
\end{table}

\begin{table}[H]
\centering
\caption{Vertex and edge chromatic numbers of Archimedean graphs}
\vspace{5pt}
\label{tab:archimedean-chrom-nums}
\begin{tabular}{|l|c|c|c|}
\hline
Archimedean & $\chi(G)$ & $\chi'(G)$ & $\chi''(G)$ \\
\hline\hline
cuboctahedron & 3 & 4 & 5 \\
\hline
icosidodecahedron & 3 & 4 & 5 \\
\hline
rhombicosidodecahedron & 3 & 4 & 5 \\
\hline
rhombicuboctahedron & 3 & 4 & 5 \\
\hline
snub cube & 3 & 5 & 6 \\
\hline
snub dodecahedron & 4 & 5 & 6 \\
\hline
truncated cube & 3 & 3 & 4 \\
\hline
truncated cuboctahedron & 2 & 3 & 4 \\
\hline
truncated dodecahedron & 3 & 3 & 4 \\
\hline
truncated icosahedron & 3 & 3 & 4 \\
\hline
truncated icosidodecahedron & 2 & 3 & 4 \\
\hline
truncated octahedron & 2 & 3 & 4 \\
\hline
truncated tetrahedron & 3 & 3 & 4 \\
\hline
\end{tabular}
\end{table}

Note, that from the tables above, we see that indeed all the above graphs have $\chi(G)$ at most 4. This is due to the famous \textit{Four Color Theorem} \cite{appelhaken76} for planar graphs.

\vspace{5pt}
\todo[inline]{TODO: Mention the Brook's theorem}

As a consequence of \textit{Vizing's theorem} \cite{misra92}, for every graph $G$ with maximum degree $\Delta(G)$, we have $\Delta(G) \leq \chi'(G) \leq \Delta(G) + 1$. This implies two classes of graphs. Class one are graphs s.t. $\chi'(G) = \Delta(G)$. Class two are then graphs s.t. $\chi'(G) = \Delta(G) + 1$. What class are graphs of Platonic and Archimedean solids?

Let us compare the degrees at each vertex of the solids as shown in tables \ref{tab:platonic-basic-props} and \ref{tab:archimedean-basic-props} with their calculated chromatic indices in the tables above. We can observe, that all the solids are of Vizing class one. Note that this is not the case for all planar graphs. In fact, there exist planar graphs with $\Delta(G)$ from 2 up to 5 such that they are class two.

Similarly, for total coloring, Vizing's conjecture \cite{vizing68} states, that for all graphs, we have $\Delta(G) + 1 \leq \chi''(G) \leq \Delta(G) + 2$. If the conjecture holds, then it again implies two classes of graphs. In the case of Platonic and Archimedean solids, it turns out, that all of them except the tetrahedron belong to class with $\chi''(G) = \Delta(G) + 1$.

\section{Some calculated chromatic polynomials}
\todo[inline]{NOTE: The following tables will most likely not be in the final thesis. Since there is not much interesting that one can see from them.}

\todo[JF]{Nebo to může jít do appendixu.}

\begin{table}[H]
\centering
\begin{tabular}{|l|p{0.5\linewidth}|}
\hline
Platonic & chromatic polynomial \\
\hline\hline
cube & $x^{8} - 12x^{7} + 66x^{6} - 214x^{5} + 441x^{4} - 572x^{3} + 423x^{2} - 133x$ \\
\hline
dodecahedron & $x^{20} - 30x^{19} + 435x^{18} - 4060x^{17} + 27393x^{16} - 142194x^{15} + 589875x^{14} - 2004600x^{13} + 5673571x^{12} - 13518806x^{11} + 27292965x^{10} - 46805540x^{9} + 68090965x^{8} - 83530946x^{7} + 85371335x^{6} - 71159652x^{5} + 46655060x^{4} - 22594964x^{3} + 7171160x^{2} - 1111968x$ \\
\hline
icosahedron & $x^{12} - 30x^{11} + 415x^{10} - 3500x^{9} + 20023x^{8} - 81622x^{7} + 241605x^{6} - 517360x^{5} + 780286x^{4} - 782108x^{3} + 463310x^{2} - 121020x$ \\
\hline
octahedron & $x^{6} - 12x^{5} + 58x^{4} - 137x^{3} + 154x^{2} - 64x$ \\
\hline
tetrahedron & $x^{4} - 6x^{3} + 11x^{2} - 6x$ \\
\hline
\end{tabular}
\end{table}
\begin{table}[H]
\centering
\begin{tabular}{|l|p{0.5\linewidth}|}
\hline
Archimedean & chromatic polynomial \\
\hline\hline
cuboctahedron & $x^{12} - 24x^{11} + 268x^{10} - 1842x^{9} + 8680x^{8} - 29516x^{7} + 74019x^{6} - 136826x^{5} + 182024x^{4} - 164656x^{3} + 90016x^{2} - 22144x$ \\
\hline
icosidodecahedron & $None$ \\
\hline
rhombicosidodecahedron & $None$ \\
\hline
rhombicuboctahedron & $None$ \\
\hline
snub cube & $None$ \\
\hline
snub dodecahedron & $None$ \\
\hline
truncated cube & $x^{24} - 36x^{23} + 622x^{22} - 6868x^{21} + 54445x^{20} - 330016x^{19} + 1590616x^{18} - 6258826x^{17} + 20483524x^{16} - 56517092x^{15} + 132781696x^{14} - 267560902x^{13} + 464751928x^{12} - 698041384x^{11} + 907685011x^{10} - 1021028578x^{9} + 990348490x^{8} - 822946048x^{7} + 579284763x^{6} - 338935770x^{5} + 159596344x^{4} - 57088336x^{3} + 13839584x^{2} - 1703168x$ \\
\hline
truncated cuboctahedron & $None$ \\
\hline
truncated dodecahedron & $None$ \\
\hline
truncated icosahedron & $None$ \\
\hline
truncated icosidodecahedron & $None$ \\
\hline
truncated octahedron & $x^{24} - 36x^{23} + 630x^{22} - 7134x^{21} + 58707x^{20} - 373816x^{19} + 1914823x^{18} - 8098890x^{17} + 28806937x^{16} - 87308340x^{15} + 227623087x^{14} - 513887650x^{13} + 1008990864x^{12} - 1726780052x^{11} + 2576178723x^{10} - 3343211267x^{9} + 3755216148x^{8} - 3618864524x^{7} + 2949553512x^{6} - 1987203924x^{5} + 1066396109x^{4} - 427989031x^{3} + 114056146x^{2} - 15071023x$ \\
\hline
truncated tetrahedron & $x^{12} - 18x^{11} + 149x^{10} - 752x^{9} + 2586x^{8} - 6408x^{7} + 11774x^{6} - 16189x^{5} + 16468x^{4} - 11869x^{3} + 5442x^{2} - 1184x$ \\
\hline
\end{tabular}
\end{table}

\todo[JF]{Případně okomentujte, co znamená None, resp. proč to nešlo spočítat.}

