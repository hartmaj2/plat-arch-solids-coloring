\chapter{Chromatic numbers and chromatic polynomials}

\section{Computing chromatic numbers}

The vertex and edge chromatic numbers can be computed in little time using \textit{SageMath} functions \cite{sagemath-chromatic-number} \cite{sagemath-chromatic-index}. Since as of time of writing this thesis, there exists no function for directly computing the total chromatic number $\chi''(G)$, the conversion to total graph mentioned in chapter \ref{chap:clring_conversions} must be used. 

The following two tables provide overview of vertex, edge and total chromatic numbers, denoted by $\chi(G)$, $\chi'(G)$ and $\chi''(G)$ respectively, for Platonic and Archimedean solids. Note that the edge chromatic number $\chi'(G)$ is also called the \textit{chromatic index}.

\begin{table}[H]
\centering
\begin{tabular}{l@{\hspace{1.5cm}}ccc}
\toprule
\textbf{Platonic} & \textbf{$\chi(G)$} & \textbf{$\chi'(G)$} & \textbf{$\chi''(G)$} \\
\midrule
cube & 2 & 3 & 4 \\
dodecahedron & 3 & 3 & 4 \\
icosahedron & 4 & 5 & 6 \\
octahedron & 3 & 4 & 5 \\
tetrahedron & 4 & 3 & 5 \\
\bottomrule
\end{tabular}
\caption{Vertex and edge chromatic numbers of Platonic graphs}
\label{tab:platonic-chrom-nums}
\end{table}

\begin{table}[H]
\centering
\begin{tabular}{l@{\hspace{1.5cm}}ccc}
\toprule
\textbf{Archimedean} & \textbf{$\chi(G)$} & \textbf{$\chi'(G)$} & \textbf{$\chi''(G)$} \\
\midrule
cuboctahedron & 3 & 4 & 5 \\
icosidodecahedron & 3 & 4 & 5 \\
rhombicosidodecahedron & 3 & 4 & 5 \\
rhombicuboctahedron & 3 & 4 & 5 \\
snub cube & 3 & 5 & 6 \\
snub dodecahedron & 4 & 5 & 6 \\
truncated cube & 3 & 3 & 4 \\
truncated cuboctahedron & 2 & 3 & 4 \\
truncated dodecahedron & 3 & 3 & 4 \\
truncated icosahedron & 3 & 3 & 4 \\
truncated icosidodecahedron & 2 & 3 & 4 \\
truncated octahedron & 2 & 3 & 4 \\
truncated tetrahedron & 3 & 3 & 4 \\
\bottomrule
\end{tabular}
\caption{Vertex and edge chromatic numbers of Archimedean graphs}
\label{tab:archimedean-chrom-nums}
\end{table}

Note, that from the tables above, we see that indeed all the above graphs have $\chi(G)$ at most 4. This is due to the famous \textit{Four Color Theorem} \cite{appelhaken76} for planar graphs.

\begin{highlight}

Another fact worth mentioning is that using the results of the \textit{Brook's theorem} \cite{brooks41}, we have that the only graph from the table above s.t. it has $\chi(G) = \Delta(G) + 1$ should be the tetrahedron which is indeed true by checking with tables \ref{tab:platonic-basic-props} and \ref{tab:archimedean-basic-props}. 

\end{highlight}

As a consequence of \textit{Vizing's theorem} \cite{misra92}, for every graph $G$ with maximum degree $\Delta(G)$, we have $\Delta(G) \leq \chi'(G) \leq \Delta(G) + 1$. This implies two classes of graphs. Class one are graphs s.t. $\chi'(G) = \Delta(G)$. Class two are then graphs s.t. $\chi'(G) = \Delta(G) + 1$. What class are graphs of Platonic and Archimedean solids?

Let us compare the degrees at each vertex of the solids as shown in tables \ref{tab:platonic-basic-props} and \ref{tab:archimedean-basic-props} with their calculated chromatic indices in the tables above. We can observe, that all the solids are of Vizing class one. Note that this is not the case for all planar graphs. In fact, there exist planar graphs with $\Delta(G)$ from 2 up to 5 such that they are class two.

Similarly, for total coloring, Vizing's conjecture \cite{vizing68} states, that for all graphs, we have $\Delta(G) + 1 \leq \chi''(G) \leq \Delta(G) + 2$. If the conjecture holds, then it again implies two classes of graphs. In the case of Platonic and Archimedean solids, it turns out, that all of them except the tetrahedron belong to class with $\chi''(G) = \Delta(G) + 1$.

\section{Computing chromatic polynomials}

\begin{highlight}

Chromatic polynomial of any graph $G=(V,E)$ can be calculated recursively using the following fact: When we fix two vertices $u$, $v$ s.t. $\{u,v\} \notin E$, we can split all colorings of $G$ into two disjoint groups. Let $d$ be the amount of colorings of $G$ in which $u$ and $v$ are colored by different color and let $s$ be the amount of colorings in which $u$ and $v$ are colored by same colors. Then the amount of all colorings $P(G,x) = d + s$. Let $G+\{u,v\}$ be graph $G$ s.t. its set of edges is $E \cup \{u,v\}$. Let $G \cdot \{u,v\}$ be the graph $G + \{u,v\}$ where the edge $\{u,v\}$ is contracted into a single vertex. Then we can see that $d = P(G + \{u,v\},x)$ and $s = P(G \cdot \{u,v\},x)$. This fact yields the following formula \cite{chartrand2019}:
\begin{equation}\label{eqn:chrom_poly_nonedge}
 P(G,x) = P(G + \{u,v\},x) + P(G \cdot \{u,v\},x) \tag{$P$}
\end{equation}

The formula above serves as the recursive case of our computation i.e. when the graph has some non-edge. In the other case, the base case, the graph has no non-edges and thus it is a complete graph $K_n$ for some $n \in \mathbb{N}$. Then the chromatic polynomial is $P(K_n,x) = x \cdot (x-1) \cdot \ldots \cdot (x-n+1)$.

\end{highlight}

Consider this example: Given the graph of tetrahedron $K_4$ and $X$ the family of proper vertex colorings. The chromatic polynomial $P_{X}(K_4,x) = x \cdot (x-1) \cdot (x-2) \cdot (x-3)$. This can be seen if we label the vertices $v_1,v_2,v_3,v_4$ and imagine coloring them sequentially in the order of their labels. We have exactly $x$ colors left to use for the first vertex. With each other vertex, we have one less color available to use. 

\begin{table}[H]
\centering
\begin{tabular}{lp{0.7\linewidth}}
\toprule
\textbf{Solid} & \textbf{Chromatic polynomial} \\
\midrule
cube & $x^{8} - 12x^{7} + 66x^{6} - 214x^{5} + 441x^{4} - 572x^{3} + 423x^{2} - 133x$ \\
octahedron & $x^{6} - 12x^{5} + 58x^{4} - 137x^{3} + 154x^{2} - 64x$ \\
tetrahedron & $x^{4} - 6x^{3} + 11x^{2} - 6x$ \\
\bottomrule
\end{tabular}
\caption{Chromatic polynomial of selected solids. For other solids, the polynomial coefficients were too large and would not print nicely.}
\label{tab:selected-chrom-polys}
\end{table}


\begin{highlight}

\subsection{Chromatic polynomial of octahedron}

We get an interesting result when applying the recursive method for computing chromatic polynomial on the graph of octahedron. Note that this graph is in fact $K_{2,2,2}$, a complete tripartite graph. Let $A$ be one of the partitions and $u,v \in A$, then we have that $\{u,v\} \notin E$ but for every $w \in V \setminus A$, $\{u,w\} \in E$ and also $\{v,w\} \in E$. From this follows, that we have exactly one non-edge per partition.

Let us now analyze the recursive method using formula \ref{eqn:chrom_poly_nonedge} on octahedron. Notice, that w.l.o.g. we can start the recursion with any non-edge, because the partitions are indistinguishable. In fact, this argument can be generalized to every step of the recursion. This means that the resulting graph will be the same independent of the order, in which we pick the non-edges, to add or identify.

So the only parameter that really matters is how many steps the recursive algorithm will take, until we reach the base case. Since we have exactly $3$ partitions, and thus $3$ non edges, the recursion depth will also be $3$.

Now each branch of the recursion is simply a sequence of $3$ operations which can be either and addition of an edge, or identification of two vertices. Let us denote the edge addition operation as $\alpha$ and identification operation by $\iota$. Notice that each $\iota$ reduces the number of vertices of the resulting graph by $1$. Let $G_{o_1 \ldots o_n}$ be the graph resulting from $G$ when operations $o_1, \ldots o_n$ are performed on it in the corresponding order. Then for example $|V(G_{\alpha \iota\alpha})| = |V(G)| -1$ because the operation sequence contains exactly one of $\iota$. Since after $3$ steps, the graph has no more non-edges, we know that it is a complete graph $K_{|V(G)|-1}$. 

\end{highlight}